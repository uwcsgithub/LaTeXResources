\documentclass{article}

\usepackage{hyperref}
\usepackage{amsmath}
\usepackage{fancyvrb}
\usepackage{float}

\usepackage[margin = 0.7in]{geometry}
\hyphenpenalty = 10000


\title{Beginner's Guide to LaTeX}
\author{}
\date{}

\begin{document}
\maketitle
\clearpage

\section*{Why LaTeX?}
LaTeX makes beautiful documents, especially for mathematics. It is far more
powerful than your standard text editor, and can be extended even further with
packages for everything you can think of. 

But how does it work? \\
You write your document in \texttt{plain text} with commands that describe its
strucutre and meaning. Th \LaTeX proogram then processes your text and commands
to produce a formatted document... a lot like HTML and CSS! 

\section*{Getting Started}
A minimal \LaTeX document requires a document class, as well as begginning and
end tags (delimiters?)

\begin{verbatim}
    \documentclass{article}
    \begin{document}
        Your first document!
        % The rest of your content goes here
    \end{document}
\end{verbatim}

All commands start with a backslash ($\backslash$). \\
Every document starts with a \verb+\documentclass+ command. \\
The argument in curly braces (\{ \}) tells \LaTeX what kind of document we are
creating. In this case, it is an \texttt{article}.\\
A percent sign (\%) starts a \textit{comment} - \LaTeX ignores all characters
after a \% on a given line. 

\subsection*{Getting Started with Overleaf}
Overleaf is a website for creating \LaTeX documents. It will "compile" your
\LaTeX and automatically show you the results. 

% TO DO: CREATE EXAMPLE DOC
Click \href{google.com}{here} to open the example document in Overleaf.

\textbf{EX 1 and EX2 here? See source file for details}
% EX 1: Add plain, unstyled text between \begin{document} and \end{document} and
% compile. Maybe play with whitespace and see how it's trimmed? 

% EX 2: Special characters in text, escape character, quatation marks (`' and ``
% ''), etc. We could possibly provide a block of text here and a solution so
% that we can guarantee all the necessary things are covered? 

\subsection*{Handling Errors}
\LaTeX can run into issues when compiling your document, just like any
programming language. When this happens, it stops and throws an error, and will
not produce any output until this error is fixed. \\
For example, if you misspell a command, say \verb+\textbf+ as \verb+\txetbf+,
\LaTeX will stop with an ``Undefined control sequence'' error, as ``txetbf'' is
not a command it knows.

In Overleaf, the errors panel is accessed by clicking the page button next to
``Recompile''. If you have errors, there will be a red icon on this button, and
the errors will be listed. 

If you encounter an error, don't panic! Look at the \LaTeX you have written and
try to figure out where it went wrong. If you have multiple errors, start with
the first one - it's possible this is leading to all of the others. 

\subsection*{Typesetting Maths}
As you saw earlier, dollar signs are special characters in \LaTeX. This is
because they're used to mark mathematics in text (inline notation).

For example, ``Let $a$ and $b$ be distinct positive integers, and let $c = a - b
+ 1$'' can be written as ``\texttt{Let \$a\$ and \$b\$ be distinct positive
intergers, and let \$c = a - b + 1\$}'' Dollar signs are always used in pairs -
once to begin and once to end the inline equation. Spacing is also automatically
handled; any spaces you use will be ignored. 

\textbf{Here's some of the basic notation in maths:}
\begin{itemize}
    \item Use a caret (\verb+^+) for superscripts and an underscore (\verb+_+)
    for subscripts
    \begin{itemize}
        \item \verb|$y = c _ 2 x^2 + c_1 x + c_0$|
        \item Which compiles to $y = c _ 2 x^2 + c_1 x + c_0$
    \end{itemize}
    \item Use curly braces (\{ \}) to group superscripts and subscripts. 
    \begin{itemize}
        \item \verb|$F_n = F_{n - 1} + F_{n - 2}$|
        \item Which compiles to $F_n = F_{n - 1} + F_{n - 2}$
    \end{itemize}
    \item There are commands for Greek letters, beginning with a backslash
    followed by the name of the character
    \begin{itemize}
        \item \verb|$\Omega = \sum_{k = 1}^{n} \omega_k$|
        \item Which compiles to $\Omega = \sum_{k = 1}^{n} \omega_k$
    \end{itemize}
\end{itemize}

For larger, more complicated equations, we can display these on their own line
with \textit{display} delimiters. For these, you can use either
\texttt{$\backslash$[ ... $\backslash$]}  or \verb+\begin{equation} ...\end{equation}+

For example,
\begin{figure}[H]
    \centering
    \begin{BVerbatim}
        \[ x = \frac{-b \pm \sqrt{b^2 - 4ac}}{2a} \] 
    \end{BVerbatim}
\end{figure}



Will display as
\[ x = \frac{-b \pm \sqrt{b^2 - 4ac}}{2a} \] 

\textbf{EX 3 here?}
% EX 3: maybe some maths problems/simple proofs to typeset their answers to. 

\subsection*{Environments}
\texttt{equation} is an environment - it gives a command context for the output it should be producing. 
Inline equations can be denoted by \verb+\begin{math} ...\end{math}+, and display differently to \verb+\begin{equation} ...\end{equation}+. For example, the summarion sign $\sum$ will be larger in a display equation than an inline one.
The \verb+\begin+ and \verb+\end+ commands are used to create many different environments. For example, the \texttt{itemize} and \texttt{enumerate} environments create bulleted and numbered lists respectively. 

\begin{figure}[H]
    \centering
    \begin{BVerbatim}
        \begin{itemize}
            \item Item 1 
            \item Item 2
        \end{itemize}
    \end{BVerbatim}
\end{figure}

\begin{itemize}
    \item Item 1 
    \item Item 2
\end{itemize}

\subsection*{Packages}
So far, everything we have used has been built into \LaTeX. However, \LaTeX has libraries, called \textit{packages}, of extra commands and environments.\\
We have to load each of the packages we want to use with a \verb|\usepackage| command in the preamble (before the \verb|begin{document}| command). 

\subsubsection*{amsmath}
The American Mathematical Society has a package called \texttt{amsmath}, imported with \verb|\usepackage{amsmath}| command in the preamble. This package provides many useful features, and the documentation for it is \href{https://www.ams.org/arc/tex/amsmath/amsldoc.pdf}{here}. 
Some examples of uses of \texttt{amsmath} are below: 

\begin{itemize}
    \item 
    \begin{verbatim}
        \begin{equation*}
            \Omega = \sum_{k - 1}^{n} \omega_k
        \end{equation*}
    \end{verbatim}
    \begin{itemize}
        \item Using \texttt{equation*} will produce unnumbered equations. 
    \end{itemize}
    \item 
    \begin{verbatim}
        \operatorname{My Operator}
    \end{verbatim}
    \begin{itemize}
        \item The \verb|\operatorname| command allows for mathematical operators not within the package to be displayed correctly.
    \end{itemize}
\end{itemize}

It can also be used to align a sequence of equations at the equals sign:

\begin{figure}[H]
\centering
\begin{BVerbatim}
    \begin{align*}
        (x + 1)^3 &= (x + 1)(x + 1)(x + 1) \\
                  &= (x + 1)(x^2 + 2x + 1) \\
                  &= x^3 + 3x^2 + 3x + 1
    \end{align*}
\end{BVerbatim}
\end{figure}

\begin{align*}
    (x + 1)^3 &= (x + 1)(x + 1)(x + 1) \\
                &= (x + 1)(x^2 + 2x + 1) \\
                &= x^3 + 3x^2 + 3x + 1
\end{align*}

The ampersand (\&) separates the left column (before the $=$) and the right column (after the $=$).\\
The double backslash ($\backslash \backslash$) starts a new line.

\textbf{EX 4 here?}
% EX 4: A more complicated exercise with multi-line aligned proofs etc? 
\end{document}
